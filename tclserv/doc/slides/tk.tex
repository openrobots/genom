%%
%% Anthony Mallet - Sun Dec  8 2002
%% Copyright (C) 2002 LAAS/CNRS 
%%
%% $Id$
%%
\documentclass[a4paper,landscape,smooth]{show}
\usepackage{depend,aeguill,cartouche}
\usepackage[dpi=100,resample]{docfig}
\usepackage[T1]{fontenc}
\usepackage[latin1]{inputenc}

\context{\vfill}
\title{Introduction à \Huge Tk}
\author{Anthony Mallet\\ basé sur le cours de E.J. Friedman-Hill}
\date{6 décembre 2002}

\newcommand{\tclex}[2]{\texttt{#1}\\$\rightarrow$ \texttt{#2}}

\begin{document}
\maketitle

% =======================================================================

\begin{part}{Introduction}{Le « toolkit » Tk}
   \vfill
   \begin{bitemize}{color@bulle}
      \item {\bf Le problème} : Construire une application avec une
	 interface est complexe. D'autant plus si la portabilité est en
	 jeu (multi-plateforme).

      \item {\bf La mauvaise solution} : 
	 C++, toolkits orientés objet, Java AWT, ...
	 Peu de bénéfices : nécéssité de programmer a un niveau assez
	 bas.

      \item {\bf La bonne solution} : 
	 Élever le niveau de programmation. Créer des interfaces à l'aide
	 de scripts Tcl.
   \end{bitemize}
   \vfill
   (Argumentaire de {\bf E.J. Friedman-Hill})
   \vfill
\end{part}

% -----------------------------------------------------------------------

\begin{tslide}{Exemple}
   \vfill
   {\tt
   listbox .list -yscroll ".scroll set"\\
   pack .list -side left\\
\\
   scrollbar .scroll -command ".list yview"\\
   pack .scroll -side right -fill y\\
\\
   wm title . "File Browser"\\
   foreach i [lsort [glob * .*]] \{\\
\hspace*{1cm}.list insert end \$i\\ \}\\
   }
   \vfill
   \vbox to0pt{\vss\hfill\docfig[width=0.3\linewidth]{fig/file-browser.jpg}}
\end{tslide}

% -----------------------------------------------------------------------

\begin{tslide}{Éléments d'une interface Tk}
   \vfill
   \begin{bitemize}{color@bulle}
      \item Commandes Tcl supplémentaires : 
	 \begin{itemize}
	    \item Créer des objets {\em à la} Motif (multi-platorme).
            \item Agencer les objets. 
            \item Lier des évènements à des scripts Tcl. 
            \item Manipuler la selection, le focus, le window manager,
		  etc. 
	 \end{itemize}

      \item Bibliothèque de fonctions C :
	 \begin{itemize}
	    \item Créer de nouvelles classes. 
	    \item Créer de nouveau gestionnaires de « géométrie ». 
	 \end{itemize}
   \end{bitemize}
   \vfill
\end{tslide}

% -----------------------------------------------------------------------

\begin{tslide}{Classe de base : le {\em widget}}
   \vfill
   {\bf Widgets}
   \begin{bitemize}{color@bulle}
      \item Windowed gadgets $\rightarrow$ gadgets fenêtrés
      \item gadget (n.)\\
	 1. a small mechanical device or appliance.\\
	 2. any object that is interesting for its ingenuity or novelty
	    rather than for its practical use.\\

	 [ perhaps from French « gachette » : lock catch, trigger,
	    diminutive of gache staple ].
   \end{bitemize}
   \vfill
\end{tslide}

% =======================================================================

\begin{part}{Tk}{Widgets}
   \vfill\small
   \begin{bitemize}{color@bulle}
      \item Toplevel
      \item Frame
      \item Menu, Menubutton
      \item Label, Message
      \item Button, Checkbutton, Radiobutton
      \item Entry, Text
      \item Listbox
      \item Scale, Scrollbar
      \item Canvas
   \end{bitemize}
   \vfill
\end{part}

% -----------------------------------------------------------------------

\begin{tslide}{Hierarchie de widgets}
   \vfill\begin{center}
      \docfig[width=0.9\linewidth,psfrag]{tk-hierarchy.fig}
   \end{center}\vfill
\end{tslide}

% -----------------------------------------------------------------------

\begin{tslide}{Création de widgets}
   \vfill
   \begin{bitemize}{color@bulle}
      \item Chaque widget a une classse : button, listbox, scrollbar,
	    etc. 
      \item Une commande de creation pour chaque classe, utilisée pour
	    créer des instances.
	    
	    {\tt button .a.b -text Quit -command exit}\\
	    {\tt scrollbar .x -orient horizontal}
   \end{bitemize}
   \vfill
\end{tslide}

% -----------------------------------------------------------------------

\begin{tslide}{Configuration}
   \vfill
   Options de configuration défines par la classse. Exemple pour la
   classe Bouton:

   {\tt
      -activebackground -disabledforeground -justify -underline \\
      -activeforeground -font -padx -width \\
      -anchor -foreground -pady -wraplength \\
      -background -height -relief \\
      -bitmap -highlightbackground -state \\
      -borderwidth -highlightcolor -takefocus \\
      -command -highlightthickness -text \\
      -cursor -image -textvariable \\
}

   \vfill
   Si certaines options ne sont pas spécifiées explicitement, elles
   viennent du fichier {\tt .Xdefaults} (Unix).
   \vfill
\end{tslide}

% -----------------------------------------------------------------------

\begin{tslide}{Commandes associées aux widgets}
   \vfill
   Une commande Tcl pour chaque widget, nommée d'après le nom du widget.

   Permet de reconfigurer et de manipuler le widget~:

   {\tt
      button .a.b\\
      .a.b configure -relief sunken\\
      .a.b flash\\
\\
      scrollbar .x\\
      .x set 0.2 0.7\\
      .x get\\

      .a.b configure -foreground\\
      $\rightarrow$ -foreground foreground Foreground blue blue}

   La commande est supprimée quand le widget est détruit
   \vfill
\end{tslide}

% -----------------------------------------------------------------------

\begin{tslide}{Placement}
   \vfill
   {\bf Geometry management}
   \begin{bitemize}{color@bulle}
      \item {\tt place}

	 Placement absolu du widget par rapport à son parent. Spécification
	 des coordonnées.

      \item {\tt pack}
	 
	 Arrange les widgets en fonction de la dimension du parent et des
	 autres widgets au même niveau.

      \item {\tt grid}

	 Arrange les widgets dans un tableau à deux dimensions.
	 
      \item {\tt frame}
      
	 Pour créer des arrangements plus complexes.
   \end{bitemize}
   \vfill
\end{tslide}

% -----------------------------------------------------------------------

\begin{tslide}{Connection à tcl}
   \vfill
   \begin{bitemize}{color@bulle}
      \item Scripts Tcl pour faire communiquer les widgets avec
	    l'application, ou les autres widgets.

	    {\tt button .a.b -command exit}\\
	    {\tt scrollbar .s -command ".text yview"}

      \item L'application utilise les commandes des widgets pour
	    communiquer avec l'interface.
   \end{bitemize}
   \vfill
\end{tslide}

% -----------------------------------------------------------------------

\begin{tslide}{Évènements}
   \vfill
   \begin{bitemize}{color@bulle}
      \item Associer des scripts Tcl aux évènements utilisateur : 

         {\tt bind .b <Control-h> \{puts "control-h"\}}

      \item Nommage des évènements : 

	 {\tt <Double-Control-ButtonPress-1> \\
	       <KeyPress> \\
	       a}

      \item Substitutions : 

	 {\tt bind .c <B1-Motion> \{move \%x \%y\}\\
	       bind .t <KeyPress> \{insert \%A\}}
   \end{bitemize}
   \vfill
\end{tslide}


% -----------------------------------------------------------------------

\begin{tslide}{Autres commandes}
   \vfill
   \begin{bitemize}{color@bulle}
      \item Selection (X11): 

	 {\tt selection get}

      \item Envoyer des commandes à d'autres applications Tk (X11) :

	 {\tt send tgdb "break tkEval.c:200" }

      \item Informations sur les fenêtres :

	 {\tt winfo width .x\\
	       winfo children .\\
	       winfo containing \$x \$y}
   \end{bitemize}
   \vfill
\end{tslide}

% -----------------------------------------------------------------------

\vfill\eject\null\vfill\eject

\end{document}
