%
% Copyright (c) 2001 LAAS/CNRS                        --  Thu Nov  8 2001
% All rights reserved.                                     Anthony Mallet
%
% This document is a translation of the French documentation of GenoM,
% originally written by Sara Fleury and Matthieu Herrb.
%
% Redistribution  and  use in source   and binary forms,  with or without
% modification, are permitted provided that  the following conditions are
% met:
%
%   1. Redistributions  of  source code must  retain  the above copyright
%      notice, this list of conditions and the following disclaimer.
%   2. Redistributions in binary form must  reproduce the above copyright
%      notice,  this list of  conditions and  the following disclaimer in
%      the  documentation   and/or  other  materials   provided with  the
%      distribution.
%
% THIS SOFTWARE IS PROVIDED BY THE  AUTHOR AND CONTRIBUTORS ``AS IS'' AND
% ANY  EXPRESS OR IMPLIED WARRANTIES, INCLUDING,  BUT NOT LIMITED TO, THE
% IMPLIED WARRANTIES   OF MERCHANTABILITY AND  FITNESS  FOR  A PARTICULAR
% PURPOSE ARE DISCLAIMED.  IN NO  EVENT SHALL THE AUTHOR OR  CONTRIBUTORS
% BE LIABLE FOR ANY DIRECT, INDIRECT,  INCIDENTAL, SPECIAL, EXEMPLARY, OR
% CONSEQUENTIAL DAMAGES (INCLUDING,  BUT  NOT LIMITED TO, PROCUREMENT  OF
% SUBSTITUTE  GOODS OR SERVICES;  LOSS   OF  USE,  DATA, OR PROFITS;   OR
% BUSINESS  INTERRUPTION) HOWEVER CAUSED AND  ON ANY THEORY OF LIABILITY,
% WHETHER IN CONTRACT, STRICT LIABILITY, OR TORT (INCLUDING NEGLIGENCE OR
% OTHERWISE) ARISING IN ANY WAY OUT OF THE  USE OF THIS SOFTWARE, EVEN IF
% ADVISED OF THE POSSIBILITY OF SUCH DAMAGE.
%
% $Id$
%

Communication between modules is based  on two libraries: \texttt{posterLib}
for posters and \texttt{csLib} for requests.  This appendix presents a short
overview of the functions of composing these libraries.

\section{Posters and \texttt{posterLib}}

Posters are  shared memory segments, that  can be written by their owners
and read by  any process.  The  basic poster handling functions and their
prototypes are:


\begin{center}\small\begin{tabularx}{\linewidth}{|l|X|}
\hline
name		& description \\
\hline
\tt posterCreate & poster creation \\
\tt posterFind	& look for a poster id given its name\\
\tt posterWrite	& write into a poster \\
\tt posterRead  & read a poster \\
\tt posterAddr  & get the poster address \\
\tt posterTake	& take the poster semaphore  \\
\tt posterGive	& give the poster semaphore \\
\tt posterIoctl & query information about the poster (date, \ldots) \\
\tt posterForget & forget about previously found poster \\
\tt posterDelete & delete a poster \\
\tt posterShow & display the state of every poster \\
\tt posterMemCreate & create a poster at the given address \\
\hline
\end{tabularx}\end{center}

\begin{center}\small\begin{tabularx}{\linewidth}{|llX|}
\hline
\tt STATUS  & \tt posterCreate & \tt (const char *name,      int size,
	                        POSTER\_ID *id); \\
\tt STATUS  & \tt posterFind   & \tt (const char *name,      POSTER\_ID *id); \\
\tt int     & \tt posterWrite  & \tt (POSTER\_ID id,   int offset,
	    		        void *buf,         int size); \\
\tt int     & \tt posterRead   & \tt (POSTER\_ID id,   int offset, 
	                        void *buf,     int size); \\
\tt void *  & \tt posterAddr   & \tt (POSTER\_ID id);  \\
\tt STATUS  & \tt posterTake   & \tt (POSTER\_ID id,   POSTER\_WRITE); \\
	    & 	or    	       & \tt (POSTER\_ID id,   POSTER\_READ); \\
\tt STATUS  & \tt posterGive   & \tt (POSTER\_ID id);   \\
\tt STATUS  & \tt posterIoctl  & \tt (POSTER\_ID id, int code, void *parg); \\
\tt STATUS  & \tt posterForget & \tt (POSTER\_ID id);  \\
\tt STATUS  & \tt posterDelete & \tt (POSTER\_ID id);  \\
\tt STATUS  & \tt posterShow   & \tt (void);  \\
\tt STATUS  & \tt posterMemCreate & \tt (const char *name,  int busSpace,
				void *pPool, int size, POSTER\_ID *id); \\
& & {\footnotesize (busSpace: POSTER\_LOCAL\_MEM,
	    POSTER\_SM\_MEM, POSTER\_VME\_A32, POSTER\_VME\_A24)} \\
\hline
\end{tabularx}\end{center}


% =======================================================================
\section{Requests and \texttt{csLib}}

Communication between modules is based  on the client/server library 
\texttt{csLib}. Messages   (requests and replies)  are   held in mailboxes  (ring
buffers).

\texttt{csLib} has the following properties:
\begin{itemize}
\item Both the sending of requests and the reception of replies can be
done in non-blocking mode.
\item The reception of a request is associated to the execution of a 
\texttt{C} function.
\item A request generates at most two replies.
\item There is no dynamic memory allocation.
\item It runs under Unix (and VxWorks).
\end{itemize}

The client side provides the following functions:

\begin{center}\small\begin{tabularx}{\linewidth}{|l|X|}
\hline
name		& description \\
\hline
\tt csClientInit 	& \\
\tt csClientRqstSend 	& Send a request\\
\tt csClientReplyRcv 	& Receive a reply\\
\hline
\end{tabularx}\end{center}


% =======================================================================
\section{Execution}

\subsubsection{Unix} Launch \texttt{h2 init}.

To  access   remote posters   (on  another  machine),  define    the 
\texttt{POSTER\_HOST} environment  variable  with the  name  of the \emph{server}
machine (there can be only one, \emph{i.e.} all the posters must be hosted
by the same machine).

The  poster server \texttt{posterServ} is  launched by \texttt{h2 init}, except
if \texttt{POSTER\_HOST} is defined.

