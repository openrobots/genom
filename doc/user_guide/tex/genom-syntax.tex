%
% Copyright (c) 2001 LAAS/CNRS                        --  Wed Nov  7 2001
% All rights reserved.                                   Florent Lamiraux
%
% This document is a translation of the French documentation of GenoM,
% originally written by Sara Fleury and Matthieu Herrb.
%
% Redistribution  and  use in source   and binary forms,  with or without
% modification, are permitted provided that  the following conditions are
% met:
%
%   1. Redistributions  of  source code must  retain  the above copyright
%      notice, this list of conditions and the following disclaimer.
%   2. Redistributions in binary form must  reproduce the above copyright
%      notice,  this list of  conditions and  the following disclaimer in
%      the  documentation   and/or  other  materials   provided with  the
%      distribution.
%
% THIS SOFTWARE IS PROVIDED BY THE  AUTHOR AND CONTRIBUTORS ``AS IS'' AND
% ANY  EXPRESS OR IMPLIED WARRANTIES, INCLUDING,  BUT NOT LIMITED TO, THE
% IMPLIED WARRANTIES   OF MERCHANTABILITY AND  FITNESS  FOR  A PARTICULAR
% PURPOSE ARE DISCLAIMED.  IN NO  EVENT SHALL THE AUTHOR OR  CONTRIBUTORS
% BE LIABLE FOR ANY DIRECT, INDIRECT,  INCIDENTAL, SPECIAL, EXEMPLARY, OR
% CONSEQUENTIAL DAMAGES (INCLUDING,  BUT  NOT LIMITED TO, PROCUREMENT  OF
% SUBSTITUTE  GOODS OR SERVICES;  LOSS   OF  USE,  DATA, OR PROFITS;   OR
% BUSINESS  INTERRUPTION) HOWEVER CAUSED AND  ON ANY THEORY OF LIABILITY,
% WHETHER IN CONTRACT, STRICT LIABILITY, OR TORT (INCLUDING NEGLIGENCE OR
% OTHERWISE) ARISING IN ANY WAY OUT OF THE  USE OF THIS SOFTWARE, EVEN IF
% ADVISED OF THE POSSIBILITY OF SUCH DAMAGE.
%

This chapter gives a description of the different commands defining
a {\GenoM} module. These commands are organized into sections
corresponding to the different parts of a GenoM module.

\section{Module Declaration}
\label{sec|declaration}

\noindent
{\tt module moduleName \{ ... \};} declaration of module with name
{\tt moduleName}. \\

\noindent
The sub-commands of {\tt module} are the following:

\begin{itemize}
\item[]{\tt number:  n;} non-negative integer uniquely identifying the module in
a given control architecture.

\item[]\texttt{version:  "major.minor";} defines the version number of the 
module

\item[]\texttt{email:  email;} defines the e-mail address of
the person to contact for information about this module.

\item[]\texttt{requires:  packageOrModule1 , ..., packageOrModuleN;}
specifies the dependency of the module. When generating the module,
{\GenoM} will look for \texttt{pkgconfig} input files
\texttt{packageOrModule1.pc},...,\texttt{packageOrModuleN.pc} and set
the necessary compilation flags accordingly. 

\item[]\texttt{codels\_requires:  packageOrModule1 , ..., packageOrModuleN;}
specifies the dependency of the codels. When configuring the codels,
{\GenoM} will look for \texttt{pkgconfig} input files
\texttt{packageOrModule1.pc},...,\texttt{packageOrModuleN.pc} and set
the necessary compilation flags accordingly. 

\item[]\texttt{lang:  "c++"} or \texttt{"c"}: specifies the programming
language with which the codels are developed.

\item[]{\tt internal\_data}: specifies the name of the internal database.
The internal database is a standard C data-structure that should
include no pointer.

\item[]\texttt{clock\_rate}: specifies the rate of the internal clock
producing the system tick, on which period tasks are synchronized.
This declaration is optional. The default value, if not specified is
100 ticks per second.
\end{itemize}

\section{Requests}
\label{sec|requests}

{\tt Request requestName \{ ... \};} declaration of a request with
name {\tt requestName}.\\

\noindent
The sub-commands of {\tt Request} are the following.\\

\begin{itemize}
\item[]{\tt doc: }''\texttt{comment}''{\tt;} inline help of the request.
\item[]{\tt type:  control}, {\tt exec} or {\tt init} defines the type of the request:
  \begin{itemize}
  \item {\tt control}: for a control request; control requests do not
    trigger any specific action. they are mostly used for accessing the
    internal database (reading or changing parameters),
  \item {\tt exec}: for an execution request; execution requests trigger
    (periodic or not) actions defined into execution codels,
  \item {\tt init}: for the initialization request of the module. No
    execution request can be treated until this request has been
    called and has returned success.
  \end{itemize}

\item[]{\tt exec\_task:  execTaskName;} defines which execution task
executes the request. Applies only for {\em execution} and {\em
  initialization} tasks.

\item[]\texttt{input: name::idbRef;} defines the input of the
  request. \texttt{name} is the name of the variable that will be
  defined in the request codels. \texttt{idbRef} is the field of the
  internal database defining the type of the codels input. For instance:\\
\begin{verbatim}
module Test {
    internal_data:	    TEST_STR;
    number:		    100;
}; 

typedef struct MY_STR {
  double a1;
  double a2;
} MY_STR;

typedef struct TEST_STR {
  MY_STR myStr;
} TEST_STR;

request Req {
  ...
  input:              a1::mStr.a1;
  ...
}
\end{verbatim}
defines a request with as input a {\tt double}\\

\item[]{\tt input\_info:  defaultValue1::"doc1" , ..., defaultValueN::"docN";}
\item[]{\tt posters\_input:  STRUCT\_NAME\_1 , ..., STRUCT\_NAME\_N;}
specifies that the codels of the request need to read information on posters exporting
data-structures of types {\tt STRUCT\_NAME\_1}, ..., {\tt
  STRUCT\_NAME\_N}. {\GenoM} server part then defines functions to read
the poster with the following prototypes:\\{\small
{\tt STATUS moduleNameSTRUCT\_NAME\_1PosterRead(POSTER\_ID posterId,
  STRUCT\_NAME\_1* x);}\\
{\tt STATUS moduleNameSTRUCT\_NAME\_1PosterFind(char *posterName,
  POSTER\_ID *posterId);}\\
...\\
{\tt STATUS moduleNameSTRUCT\_NAME\_NPosterRead(POSTER\_ID posterId,
  STRUCT\_NAME\_1* x);}\\
{\tt STATUS moduleNameSTRUCT\_NAME\_NPosterFind(char *posterName,
  POSTER\_ID *posterId);}\\
}
\item[]{\tt output:  name::idbRef;} output of the request. The type of
the output is defined by field {\tt idbRef} of internal database
similarly as input.\\

\item[]{\tt codel\_control: codelName;} specifies the name of the
control codel of the request. The control codel is called before
any other codels when request is triggered. If the control codel returns {\tt ERROR}, the
request ends. The prototype of the control codels is defined by the
input and output of the request.

\item[]{\tt codel\_start: codelName;} (only for initialization and execution
tasks.) specifies the name of the start codel of the request. Called
after the control codel. The start codel can return:
\begin{itemize}
\item {\tt EXEC}: the function defined by {\tt codel\_main} is called
  at the next execution period,
\item {\tt END}: the function defined by {\tt codel\_end} is called at the
  next execution period,
\item {\tt ETHER}: the request terminates.
\end{itemize}

\item[]{\tt codel\_main: codelName;} (only for initialization and execution
tasks) specifies the name of the main codel of the request. The main
codel can return:
\begin{itemize}
\item {\tt EXEC}: the main codel is called again
  at the next execution period,
\item {\tt END}: the function defined by {\tt codel\_end} is called at the
  next execution period,
\item {\tt ETHER} the request terminates.
\end{itemize}

\item[]{\tt codel\_end:  codelName;} (only for initialization and execution
tasks) terminates request execution and returns {\tt ETHER}.

\item[]{\tt codel\_inter: codelName;} (only for initialization and execution
tasks) specifies the name of the interruption codel. The interruption
codel is called when the request is interrupted by another request
defined by field {\tt interrupt\_activity}\\

\item[]{\tt fail\_reports:  reportName1 , ..., reportNameN;}
possible error messages generated by the request. Error messages are
transformed into unique integer values stored into C macros with
names:
\begin{verbatim}
S_moduleName_reportName1, ..., S_moduleName_reportNameN
\end{verbatim}
where \texttt{moduleName} is the name of the module.

\item[]{\tt interrupt\_activity:	    RequestName1, ..., RequestNameN;}
defines the names of the requests of this module that interrupt this request when
triggered by a client. {\tt RequestName1, ..., RequestNameN} can be
replaced by {\tt all} to denote all the requests of the module.

\end{itemize}

\section{Posters}
\label{sec|posters}

{\tt poster posterName \{ ... \};} declaration of poster with name {\tt
  posterName}.\\

\noindent
The sub-commands of {\tt poster} are the following.

\begin{itemize}
\item[]{\tt update:  user} or {\tt auto;} defines how the poster is updated,
either automatically ({\tt auto}) or by the user in the codels ({\tt user})

\item[]{\tt type:  structName1, ..., structNameN;} defines the types of
the sub-fields of the poster. Can be used only with {\tt update:  user;}.

\item[]{\tt data:  name1::idbRef1 , ..., nameN::idbRefN;} defines the
types of the sub-fields of the poster by internal database fields.

\item[]{\tt exec\_task:  taskName;} defines the name of the task that updates
the poster.

\item[]{\tt codel\_poster\_create:  codelName;} defines the codel that creates
the poster. Can be used only with {\tt update:  user;}.
\end{itemize}

\section{Execution Tasks}
\label{sec|exec-task}

\begin{itemize}
\item[]{\tt exec\_task taskName \{ ... \};} defines a new execution task. The
sub-commands of an execution task declaration are the following.

\item[]{\tt period:  number;} defines the period of the task in
ticks. The number of ticks per second is defined by the
global \texttt{clock\_rate} definition of the module.

The special value \textbf{\texttt{none}} can be used to create asynchronous
execution tasks.

\item[]{\tt delay:  number;} defines the offset of the task with respect to
the beginning of the period.

\item[]{\tt priority:  number;} defines the priority of the task.

\item[]{\tt stack\_size:  size;} defines the size of the task stack.

\item[]{\tt posters\_input:  STRUCT\_NAME\_1 , ..., STRUCT\_NAME\_N;}
specifies that the codels of the task need to read information on posters exporting
data-structures of types {\tt STRUCT\_NAME\_1}, ..., {\tt
  STRUCT\_NAME\_N}. {\GenoM} server part then defines functions to read
the poster with the following prototypes:\\ {\small
{\tt STATUS moduleNameSTRUCT\_NAME\_1PosterRead(POSTER\_ID posterId,
  STRUCT\_NAME\_1* x);}\\
{\tt STATUS moduleNameSTRUCT\_NAME\_1PosterFind(char *posterName,
  POSTER\_ID *posterId);}\\
...\\
{\tt STATUS moduleNameSTRUCT\_NAME\_NPosterRead(POSTER\_ID posterId,
  STRUCT\_NAME\_1* x);}\\
{\tt STATUS moduleNameSTRUCT\_NAME\_NPosterFind(char *posterName,
  POSTER\_ID *posterId);}\\
}

\item[]{\tt codel\_task\_start:  codelName;} defines the codel called at
initialization of the task when the module is launched.

\item[]{\tt codel\_task\_end:  codelName;} defines the codel called when the task ends.

\item[]{\tt codel\_task\_main:  codelName;} defines the codel systematically
called by the task (before all the other activities).

\item[]{\tt codel\_task\_main2:  codelName;} defines the codel systematically
called by the task (after all the other activities).

\item[]\texttt{codel\_task\_wait: codelName}: defines a
``\emph{waiting}'' codel for asynchronous execution tasks, that is
called before \texttt{codel\_task\_main}.  When this codel
is called, it doesn't have the lock on the internal data structure of
the module. Its goal is to block the task, waiting on availability of
data on a file descriptor or a semaphore. 

It's not permitted to declare a \texttt{codel\_task\_wait} codel for a
periodic task.

\item[]{\tt fail\_reports:  reportName1 , ..., reportNameN;} 
possible error messages generated by the task. Error messages are
transformed into unique integer values.
\end{itemize}
