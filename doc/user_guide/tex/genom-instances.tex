As of Genom version 2.8.2, it is possible to have several instances of
the same module running on one machine. 

Genom module instances are supported in the C/C++ and Tcl bindings for
Genom.

\section{Identifying instances}

A genom module instance is identified by a string suffix added to the
module name in the mailbox name and all poster names (plus a few
internal pocolibs identifiers that are normally not used directly by
Genom modules). 

In this documentation an \textbf{instance name} of a module is the
name of the module plus the instance suffix. 
For example for module \texttt{demo}, if the instance ``\texttt{0}''
is created, the instance name is \texttt{demo0}.

Note that the suffix should be a short string choosen to keep the
instance name as a valid identifier in the languages used by
Genom. Including a word separator, such as whitespace or punctuation
characters, is going to create various, undefined behaviours. 

The \textbf{default instance} of a module, is a module launched
whitout specifying any instance. It uses the bare module name as
identifier base for its mailbox and posters.

\section{Starting a module instance}

To launch a module with an instance name,  the
\texttt{GENOM\_INSTANCE\_}\textit{module} environment variable should
be to the specified instance suffix. 

For examle, for module demo, instance \texttt{demo0} is created by:

\texttt{GENOM\_INSTANCE\_demo=0}

\section{Accessing data in a module instance}

\section{C language}

Instance aware poster access functions are generated by Genom: 

\texttt{modulePosterNameInstancePosterRead(instance\_name, data)}
reads the poster of the specified \texttt{instance\_name}.

\section{tclserv}

Load module instances as aliases : 

\texttt{lm demo as demo0}

loads the instance \texttt{demo0} of the demo module. It creates the
\texttt{demo0} name space which will allow to interact with that
instance of the module. 

More instances of the same module can be loaded too: 

\texttt{lm demo as demo1}

will load the \texttt{demo1} instance of the module and create the
corresponding namespace.

