%
% Copyright (c) 2001,2009 LAAS/CNRS                        --  Tue Oct 30 2001
% All rights reserved.                                     Anthony Mallet
%
% This document is a translation of the French documentation of GenoM,
% originally written by Sara Fleury and Matthieu Herrb.
%
% Redistribution  and  use in source   and binary forms,  with or without
% modification, are permitted provided that  the following conditions are
% met:
%
%   1. Redistributions  of  source code must  retain  the above copyright
%      notice, this list of conditions and the following disclaimer.
%   2. Redistributions in binary form must  reproduce the above copyright
%      notice,  this list of  conditions and  the following disclaimer in
%      the  documentation   and/or  other  materials   provided with  the
%      distribution.
%
% THIS SOFTWARE IS PROVIDED BY THE  AUTHOR AND CONTRIBUTORS ``AS IS'' AND
% ANY  EXPRESS OR IMPLIED WARRANTIES, INCLUDING,  BUT NOT LIMITED TO, THE
% IMPLIED WARRANTIES   OF MERCHANTABILITY AND  FITNESS  FOR  A PARTICULAR
% PURPOSE ARE DISCLAIMED.  IN NO  EVENT SHALL THE AUTHOR OR  CONTRIBUTORS
% BE LIABLE FOR ANY DIRECT, INDIRECT,  INCIDENTAL, SPECIAL, EXEMPLARY, OR
% CONSEQUENTIAL DAMAGES (INCLUDING,  BUT  NOT LIMITED TO, PROCUREMENT  OF
% SUBSTITUTE  GOODS OR SERVICES;  LOSS   OF  USE,  DATA, OR PROFITS;   OR
% BUSINESS  INTERRUPTION) HOWEVER CAUSED AND  ON ANY THEORY OF LIABILITY,
% WHETHER IN CONTRACT, STRICT LIABILITY, OR TORT (INCLUDING NEGLIGENCE OR
% OTHERWISE) ARISING IN ANY WAY OUT OF THE  USE OF THIS SOFTWARE, EVEN IF
% ADVISED OF THE POSSIBILITY OF SUCH DAMAGE.
%
% $Id$
%

Creating a new module implies writing two distinct parts: the description
of the module  (the \texttt{.gen} file)  and codels. This  chapter describes
the first part, the \texttt{.gen} file.

% -----------------------------------------------------------------------
\section{Using the XEmacs mode \texttt{genom-mode}}

It is strongly advised you use the \texttt{genom-mode} under XEmacs to write
your module.  Besides the syntactic  coloring and automatic  indentation,
this   mode defines several   commands   that create  \GenoM\  structures
(requests, posters, \ldots). It also includes on-line help.

Note that for now \texttt{genom-mode} works only with \texttt{xemacs} and not anymore
with emacs...

Commands can be accessed by three means:
\begin{itemize}
\item The (X)Emacs menu bar (\texttt{GenoM Mode Commands})
\item A pop-up menu with button 3 of the mouse
\item The following keyboard shortcuts, beginning with \texttt{C-c} (control-c)

{\small
\begin{tabular}{|l|p{10cm}|}
\hline
\tt C-c C-m & create a new \underline{\textbf{m}}odule  (\emph{first command to invoke})\\
\tt C-c C-i & \underline{\textbf{i}}mport structures from another module\\
\tt C-c C-r & create a \underline{\textbf{r}}equest \\
\tt C-c C-p & create a \underline{\textbf{p}}oster \\
\tt C-c C-e & create an \underline{\textbf{e}}xecution task\\
\hline
\tt C-c C-b & indent whole \underline{\textbf{b}}uffer \\
\tt C-c C-v & \underline{\textbf{v}}erify that every field is filled \\
\tt C-c C-d & remove optional fields that are empty (\underline{\textbf{d}}elete) \\
\hline
\tt C-c C-h \ldots & on-line \underline{\textbf{h}}elp\\
\hline
\end{tabular}}
\end{itemize}

Commands that create a \GenoM\ structure (request, poster, \ldots) prompt
you  for  a   name in  the   mini-buffer.  Additional   arguments may  be
requested, depending on the particular structure you  are creating.  Once
you  have  supplied all the  arguments,  a template  for the structure is
inserted  in the buffer.   Optional   fields  are surrounded  by   single
superior and inferior signs (\texttt{$<$} and \texttt{$>$}). Mandatory  fields
are surrounded by two  superior and inferior signs  (\texttt{$<<$}  and 
\texttt{$>>$}). You  can refer to the  on-line help to know  how to fill in these
fields.

On-line help can be requested with the \texttt{C-c C-h} key sequence,
followed by one of the following letter:

{\small
\begin{tabular}{|l|p{10cm}|}
\hline
\tt h & lists the available commands of \texttt{genom-mode}  \\
\tt m & describes \underline{\textbf{m}}odules \\
\tt r & describes \underline{\textbf{r}}equests \\
\tt p & describes \underline{\textbf{p}}osters \\
\tt e & describes \underline{\textbf{e}}xecution tasks \\
\tt i & describes structures \underline{\textbf{i}}mportation \\
\tt g & describes the module \underline{\textbf{g}}eneration  \\
\tt c & describes the \underline{\textbf{c}}odels  \\
\tt C-h & help on help (this list) \\
\hline
\end{tabular}}

Help pages are made up of four parts:

\begin{itemize}
\item \texttt{What is a \ldots ?}: general description of the \GenoM\
structure
\item \texttt{How to create a \ldots ?}: how to create the structure
\item \texttt{How to instantiate the fields ?}: how to fill in the template
of the structure
\item \texttt{Examples}: some examples.
\end{itemize}

Last,  active   zones (updated  with   the sequence  \texttt{C-button2}) are
defined for each request and each codel  of the requests. When clicking
(button 2) on these zones, the file containing the corresponding codel(s)
is visited  and the point is positioned  onto the function. 

Note that if this function has not been written yet (new codel or even
request), then \texttt{genom-mode} prompts you to insert the empty template
of the codel (or all the codels of the request if you click on the request  
head-line). For this to work, the the module must be first regenerated once.

% =======================================================================
\section{Writing a module}

A  module description contains five   parts. The five section below  will
describe these parts, and use the \texttt{demo} module as an example.
The five parts are:

\begin{enumerate}
\item Module declaration
\item \texttt{C} structures and fIDS declaration
\item Requests definition
\item Posters definition
\item Execution tasks declaration
\end{enumerate}

All the  \GenoM\ structures (module,  request, poster  and task) use  the
same syntax: a keyword, which characterizes the  structure, followed by a
name and several fields enclosed between  braces (\texttt{\{} and \texttt{\}}).
The keyword is one of \texttt{module, import  from, request, poster} or 
\texttt{exec\_task}.

In this section, optional  fields are  surrounded  by \texttt{$<$} and  
\texttt{$>$}. Mandatory fields are surrounded by  \texttt{$<<$} and \texttt{$>>$}. {\em
Optional   fields that are not  instantiated  must  be removed before the
module is generated}.

% -----------------------------------------------------------------------
\subsection{Module declaration}

A module declaration looks like this:

\begin{center}\begin{cartouche}\small\begin{verbatim}
module <<module-name>> {
    number:            <<module-number>>;
    internal_data:     <<SDI-type>>;
    version:           <"version-string">;
    email:             <"address">;
    requires:          <package-dependency> , ...;
    codels_requires:   <package-dependency> , ...;
    lang:              <c or c++>;
};
\end{verbatim}\end{cartouche}\end{center}


This part   is  mandatory  and    lets   you  choose a  name   for   your
module.  Fill-in the field  \texttt{$<<$module-name$>>$}.

The parameters are the following:

\begin{itemize}
\item \texttt{number:} is the identification number of the module. 
      It must be greater that $1000$ (small numbers are reserved for
      system) and smaller than $2^{15}=32768$. It should be unique and no other
      module should use the same one. It is used to generate unique error
      report numbers.

\item \texttt{internal\_data:} is the name of  a valid \texttt{C} type.  This
      will be the module internal database (IDS). \texttt{genom-mode}
      provides you with a default value for this field.

\item \texttt{version} is a string enclosed in double-quotes that defines
      the external version number of your module. This is used by the GNU
      build tools in various places. You should incremement it before
      releasing a new version of a module. 

\item \texttt{iface\_version:} is similar but it is the \texttt{libtool}
	version of the module, not the required package. The strategy 
	relatively to the module versioning is still not well defined and
	for now this field is rarely used.

\item \texttt{email} is the contact address for the module. This will
      be copied in some files generates by the GNU build tools. Make
      sure you use an address that will be valid in the future if you
      plan to release and distribute your module.

\item \texttt{requires:} allows mainly to list the other /GenoM/ modules
      that are imply within this \texttt{.gen} file (referenced
      structures). More generally, it allows to declare the list of
      packages (in the sens of \texttt{pkg-config}) require to generate this
      \texttt{.gen} file. 

\item \texttt{codels\_requires:} it allows to declare the list of packages
	(in the sens of \texttt{pkg-config})
	require to compile and link the codels (and only the codels).

\item \texttt{lang} sets the programming language used by the
      codels. It defaults to ``C''. ``C++'' can be used to set the
      language of the codels to C++.

\end{itemize}


% -----------------------------------------------------------------------
\subsection{\texttt{C} structures and fIDS declaration}

Note:  the  \emph{functional internal  data  structure} (fIDS, or  SDIf in
French) is  a \texttt{C} structure that  contains all the requests input and
output data,  as well as the  posters definition.  When writing a module,
this structure  will  be defined  progressively  by   adding the requests
parameters each time a new request is added: do not try to write it 
\emph{a priori}.

\subsubsection{Requests parameters, replies and posters}

These are \texttt{C} structures you should define  in \texttt{C} header files.
You can include  these headers with the \texttt{\#include} directive, as in
plain \texttt{C}.  Since these headers are parts  of the  module definition,
they should be located in the main directory  of your module, in the same
place as the \texttt{.gen} file.

These structures will be used by other modules or programs.  Thus, it is
\emph{strongly} advised you prefix their names with the name of your
module to avoid conflict and determine its origin (see the example 
\texttt{demoStruct.h} in the section~\ref{sec|session|example}).

Lastly, it is also advised you protect your headers of multiple inclusion
with the standard strategy:

\begin{center}\begin{cartouche}\small\begin{verbatim}
#ifndef FILENAME
#define FILENAME
...
#endif  /* FILENAME */
\end{verbatim}\end{cartouche}\end{center}

You must respect three rules in order to get your header files working
with \GenoM:

\begin{enumerate}
\item \textbf{Allowed C types}: \GenoM\ can parse \emph{nearly} all \texttt{C}
type declarations. The  only unknown type  is \texttt{void}  and a few other
constructions are  forbidden:  \emph{i.}   \texttt{union}s,  and  \emph{ii.}
recursive type definition as   in \texttt{typedef B A}   where \texttt{A}  is a
typedef itself. A workaround for the latter is to use a new structure:

\begin{center}\begin{cartouche}\small\begin{verbatim}
typedef struct B {
   A a;
} B;
\end{verbatim}\end{cartouche}\end{center}

\item \textbf{Limitations on pointers}: Requests parameters, replies and
posters  will  travel  between  several  processes,  outside  the module,
possibly on  another machine.  Given that,  the  notion of \emph{pointer},
\emph{address} or \emph{list} does not make  sense.  They should not appear
in this context (but  you  can use  such data  types internally in   your
codels).

\item \textbf{Alignment considerations}: The structures you define can
potentially be transferred across several  platforms.  You must be  aware
that different systems do not  align data in   the same way. In order  to
avoid problems, you should  align yourself your data  on \texttt{double}s (8
bytes). The following example illustrates this:

\begin{center}\begin{cartouche}\small\begin{verbatim}
typedef struct PILO_MOVE {
   int    percentSpeed;    /* percentage of max speed */
   int    padding;         /* ALIGNMENT */
   double distance;        /* distance to travel */
} PILO_MOVE;
\end{verbatim}\end{cartouche}\end{center}

\end{enumerate}


\subsubsection{External structures}

It is  possible to   import,  from  other  modules, external    structure
definitions. The corresponding headers are included with the 
\texttt{\#include} directive but, in that case, it must be protected in an 
\texttt{import from}    directive.  This tells \GenoM\     from which module  the
structures come, and  avoid duplication of  the functions that deal  with
these structures.

It is strongly recommended that you do not hard-code the path to the
external headers. 

The field \texttt{requires:} within the module declaration structure, will
allow \GenoM\ to automatically find out the corresponding header and
library files using \texttt{pkg-config}.

It is also possibly to specify path by hand using \GenoM\ options 
\texttt{-I<path>}, \texttt{-J<VAR>=<path>} or \texttt{-P<package>} (see
chapter~\vref{cha|generation}).

In the example  below, the module  \texttt{pilo} uses  structures defined in
the module \texttt{loco}:

\begin{center}\begin{cartouche}\small\begin{verbatim}
module pilo {
     ...
     requires:    loco;
};

import from loco {
#include "locoStruct.h"
};
#include "piloStruct.h"

typedef struct PILO {
     PILO_MOVE move;
     LOCO_REF  reference;
} PILO;
\end{verbatim}\end{cartouche}\end{center}

\subsubsection{What should (and should not) the fIDS contain?}

Requests parameters, requests replies  and almost every poster  will pass
through the fIDS: thus, they must be declared in this structure. The fIDS
is also a way to   exchange data between tasks    (or threads) inside   a
module. Conversely, data exchanged between codels of  the \emph{same} task
only do not need to be declared here (but only within the codels file).


% -----------------------------------------------------------------------
\subsection{Requests definition}

There     are  threes   types of      requests:  control, execution   and
initialization. The three types are  identified  by the field \texttt{type}
and one of the three keywords \texttt{control}, \texttt{exec} and \texttt{init}.

Examples of requests can be found in the chapter~\ref{cha|session}.

\subsubsection{Control requests}

They are defined with the keyword \texttt{control} in the field \texttt{type}:

\begin{center}\begin{cartouche}\small\begin{verbatim}
request <<request-name>> {
     doc:                    <"doc">;
     type:                   control;
     input:                  <name>::<sdi-ref>;
     input_info:             <default-val>::<"name">, ...;
     output:                 <name>::<sdi-ref>;
     codel_control:          <codel-name>;
     fail_reports:           <report-name>, ... ;
     interrupt_activity:     <exec-rqst-name>, ...;
};
\end{verbatim}\end{cartouche}\end{center}

\begin{itemize}
\item \texttt{doc} is a short string that describes the service usage.

\item \texttt{input} and \texttt{output} define respectively the input
parameter and the output parameter of the request. \texttt{name} is the name
of this variable and  \texttt{sdi-ref} the name  of the corresponding member
of the fIDS (e.g. \texttt{input: position::state.position}).

\item \texttt{input\_info} lets you define default values as well as a
comment for \emph{each}   member  of  the  \texttt{input}  structure.  This
information is used for interactive  requests invocation.

\item \texttt{codel\_control} is a codel (\texttt{C} function) which is
executed by the control task and which controls the validity of the input
parameter.

\item \texttt{fail\_reports} is a list of possible reports returned by the
control codel (the special report "OK" is always implicitly defined).

\item \texttt{interrupt\_activity} is a list of requests of \emph{this module}
that   are    declared   incompatible  with     this request.  Activities
corresponding to the listed requests will  be interrupted upon invocation
of this service.  Two special keywords \texttt{all} and \texttt{none} let  you
declare all requests (or none) to be incompatible with this one.
\end{itemize}


\subsubsection{Execution requests}

They are defined with the keyword \texttt{exec} in the field \texttt{type}. As
opposed to the control requests, those requests declare services that
will be executed and they define a few more fields:

\begin{center}\begin{cartouche}\small\begin{verbatim}
request <<request-name>> {
     doc:                   <"doc">;
     type:                  exec;
     exec_task:             <<exec-task-name>>;
     input:                 <name>::<sdi-ref>;
     input_info:            <default-val>::<"name">, ...;
     output:                <name>::<sdi-ref>;
     codel_control:         <codel-name>;
     codel_start:           <codel-name>;
     codel_main:            <codel-name>;
     codel_end:             <codel-name>;
     codel_inter:           <codel-name>;
     codel_fail:            <codel-name>;
     fail_reports:          <report-name>, ... ;
     interrupt_activity:    <exec-rqst-name>, ... ;
};
\end{verbatim}\end{cartouche}\end{center}

\begin{itemize}
\item \texttt{exec\_task} is the name of the execution task in charge of the
codels execution.

\item \texttt{posters\_input:} structure types of the posters that the
execution codels of this request will read (see \S~\vref{sec|posters}).

\item \texttt{codel\_activity\_start}, \texttt{codel\_activity\_main}, 
\texttt{codel\_activity\_end},\\ \texttt{codel\_activity\_inter} and  
\texttt{codel\_activity\_fail} are the codels of this service.  All fields are
optional, but at least one of \texttt{codel\_activity\_start}, 
\texttt{codel\_activity\_main} or \texttt{codel\_activity\_end} must be defined.
Be careful that all activities can be interrupted: do not forget to
fill-in \texttt{codel\_activity\_inter} if something must be done in such a
case.
Note that \texttt{codel\_activity\_fail} is rarely used.
Codels are further described in chapter~\ref{cha|codels}.


\item All other fields serve the same purpose  as in control  requests.
See previous paragraph for a description.

\end{itemize}


\subsubsection{Initialization request}

A special execution request is the \emph{initialization request}. It is
identified by the keyword \texttt{init} in the field \texttt{type} (all other
fields are the same as for execution requests).  This special request can
be used to perform some initialization upon module startup. There can be
at most one such request and the module will not accept to serve any
other \emph{execution} request until the \texttt{init} has been invoked.
Control requests will still be served, and can be used to set several
parameters used by the \texttt{init} request.

In  order to allow the  invocation  of the init  request  from a standard
shell (for instance as soon as the  module is spawned), \GenoM\ builds an
executable called \texttt{<module>Init} (where \texttt{<module>} is the name of
the module). This executable takes exactly as
many parameters as in the structure  declared in the  input field, in the
same order as they appear in the structure.

\subsection{Posters definition}

The  posters let you export   data, either automatically  (you don't have
anything to do) or ``by hand'' inside  a codel.  Data may  be a member of
the fIDS or not.

\subsubsection{Data from the fIDS}

\begin{center}\begin{cartouche}\small\begin{verbatim}
poster <<poster-name>> {
     update:             <<update-type>>;
     data:               <<name>>::<<sdi-ref>>, ... ;
     exec_task:          <<exec-task-name>>;
};
\end{verbatim}\end{cartouche}\end{center}

\begin{itemize}
\item \texttt{update} indicates whether the poster is updated automatically
(\texttt{auto}) or by  a codel (\texttt{user}). The  \texttt{auto} mode is usually
chosen for periodical data such as a position.

\item \texttt{data} is the list of data you wish to include in the
poster. It is given in the same way as the input and output parameters of
the requests: a name, followed by a reference to a member of the fIDS.

\item \texttt{exec\_task} is the task which owns the poster. This task is in
charge of the update of the poster for \texttt{auto} posters. Note that only
the task which owns the poster can change its content.
\end{itemize}

The data  structure  of the  poster is  a concatenation  of  the  list of
declared data.  The corresponding  \texttt{C} type  is defined by \GenoM\ in
the     file   \texttt{server/<module>Poster.h} and     its    name  is  
\texttt{<MODULE>\_<POSTER>\_POSTER\_STR} (all uppercase) where \texttt{<MODULE>} is the name
of the module and \texttt{<POSTER>} the name of the poster.

For  instance,   the \texttt{Mobile}  poster  of    the \texttt{demo}  module
(chapter~\ref{cha|session}) is defined as follow:

\begin{center}\begin{cartouche}\small\begin{verbatim}
typedef struct DEMO_MOBILE_POSTER_STR {
  DEMO_STATE_POSTER_STR state;
  double ref;
} DEMO_MOBILE_POSTER_STR;
\end{verbatim}\end{cartouche}\end{center}


\subsubsection{Other data}

Data exported by posters  are not necessarily members  of the fIDS.  This
can be the case if \emph{i.} data structures are big: it is not advised to
put  them in   the fIDS and   copy them  several times, \emph{ii.}  data
structures do not have a predefined size, as for lists for instance.

For this kind of posters, two new fields are defined:

\begin{center}\begin{cartouche}\small\begin{verbatim}
poster <<poster-name>> {
     update:               user;
     type:                 <<name>>::<<type-name>>, ...;
     exec_task:            <<exec-task-name>>; 
     codel_poster_create:  <codel>;
};
\end{verbatim}\end{cartouche}\end{center}

\begin{itemize}
\item \texttt{type} is the name of the \texttt{C} type of the data structure.
\item \texttt{codel\_poster\_create} optionally designates the name of a \texttt{C}
function which is   used to create the  poster  structure. If  it is  not
given,  the  module performs the memory  allocation  by itself, using the
size of the  given \texttt{C} type.
\end{itemize}


\subsection{Execution tasks declaration}

\begin{center}\begin{cartouche}\small\begin{verbatim}
exec_task <<exec-task-name>> {
     period:             <number>;
     delay:              <number>;
     priority:           <<number>>;
     stack_size:         <<number>>;
     codel_task_start:   <codel>;
     codel_task_end:     <codel>;
     codel_task_main:    <codel>;
     codel_task_main2:   <codel>;
     posters_input:      <poster-type>, ...;
     error_reports:      <report-name>, ... ;
};
\end{verbatim}%
\end{cartouche}\end{center}

\begin{itemize}
\item \texttt{period} (optional) defines a periodical task. The period is
given as an integer number in \texttt{ticks} (at the moment, a tick is $5ms$
under VxWorks and $10ms$ under Unix). \emph{Remark:} previously the period was to be 
\emph{a divisor or a multiple} of $20$. (e.g. 1, 2, 4, 5, 10, 20, 40, 60,
\ldots. This constraint has been raised.

\item \texttt{delay} (optional integer in \emph{ticks}). All periodical
tasks with the same period  will wake up at the  same time. The delay can
be used to delay the waking up of a particular task  by the amount of ticks 
specified. \texttt{delay} can be \texttt{none} for  a-periodical tasks. Pertinent only on real-time OS.

\item \texttt{priority} is used by the scheduler of the operating system. It
is an integer  between $0$ (highest) and  $255$ (lowest). Priorities must
be used to make  sure that tasks  with strong real-time constraints  will
match their requirements.  A common strategy is to use a priority roughly
``proportional to the inverse'' of the period. Pertinent only on real-time OS.

\item \texttt{stack\_size} is the size (in bytes) of the stack for this
task. The size you  need  depends essentially on   the size of the  local
variables you  use.  A   stack which  is \emph{too small} will   produce
unpredictable results, so be sure  to largely \textbf{overestimate} what you
need.   A  good choice is  usually   $20.000$  bytes. Note that under
Unix, stack size are not used at this time (the stack is grown dynamically).  Pertinent only on real-time OS.

\item \texttt{codel\_task\_start} is the initialization codel. It is called only
once, just before the module is ready to answer requests.  It can be used
to initialize internal variables (see also  the \emph{init request}, which
can be used if the initialization requires user inputs).

\item \texttt{codel\_task\_end} is the symmetric of the \texttt{codel\_task\_start}. It
is called once, just before the module exits.

\item \texttt{codel\_task\_main} is a \emph{permanent} codel. It is executed each time
the execution tasks wakes  up (\emph{before} all the other activites). Thus, for  a  periodical task, it is  also
periodical.

\item \texttt{codel\_task\_main2} is a \emph{permanent} codel. It is executed each time
the execution tasks wakes  up (\emph{after} all the other activites). Thus, for  a  periodical task, it is  also
periodical.

\item \texttt{posters\_input:} structure types of the posters that the
execution codels of this task will read (see \S~\vref{sec|posters}).

\item \texttt{error\_reports} is a list of reports that can be reported by the
permanent activity \texttt{codel\_task\_main}. Since  this activity does not belong to
a request, its  reports are  stored in  the  \emph{control poster} of  the
module.
\end{itemize}
