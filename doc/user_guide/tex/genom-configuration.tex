%
% Copyright (c) 2001-2005 LAAS/CNRS                   --  Sat Nov 19 2005
% All rights reserved.                                     Sara Fleury
%
% This document is a translation of the French documentation of GenoM,
% originally written by Sara Fleury and Matthieu Herrb.
%
% Redistribution  and  use in source   and binary forms,  with or without
% modification, are permitted provided that  the following conditions are
% met:
%
%   1. Redistributions  of  source code must  retain  the above copyright
%      notice, this list of conditions and the following disclaimer.
%   2. Redistributions in binary form must  reproduce the above copyright
%      notice,  this list of  conditions and  the following disclaimer in
%      the  documentation   and/or  other  materials   provided with  the
%      distribution.
%
% THIS SOFTWARE IS PROVIDED BY THE  AUTHOR AND CONTRIBUTORS ``AS IS'' AND
% ANY  EXPRESS OR IMPLIED WARRANTIES, INCLUDING,  BUT NOT LIMITED TO, THE
% IMPLIED WARRANTIES   OF MERCHANTABILITY AND  FITNESS  FOR  A PARTICULAR
% PURPOSE ARE DISCLAIMED.  IN NO  EVENT SHALL THE AUTHOR OR  CONTRIBUTORS
% BE LIABLE FOR ANY DIRECT, INDIRECT,  INCIDENTAL, SPECIAL, EXEMPLARY, OR
% CONSEQUENTIAL DAMAGES (INCLUDING,  BUT  NOT LIMITED TO, PROCUREMENT  OF
% SUBSTITUTE  GOODS OR SERVICES;  LOSS   OF  USE,  DATA, OR PROFITS;   OR
% BUSINESS  INTERRUPTION) HOWEVER CAUSED AND  ON ANY THEORY OF LIABILITY,
% WHETHER IN CONTRACT, STRICT LIABILITY, OR TORT (INCLUDING NEGLIGENCE OR
% OTHERWISE) ARISING IN ANY WAY OUT OF THE  USE OF THIS SOFTWARE, EVEN IF
% ADVISED OF THE POSSIBILITY OF SUCH DAMAGE.
%
% $Id$
%

% -----------------------------------------------------------------------

\section{Requirements}
\label{sec|configuration|requirements}

\subsection{Operating systems}


\subsection{External tools}

/GenoM/ needs the following tools.
They are generally available on most systems. If not, download
them from the indicatedXXX web sites. 

\begin{itemize}
\item {\tt autoconf} version 2.59 or later
\item {\tt automake} version 1.8 or later
\item {\tt GNU make} version 3.79 or later
\item {\tt pkgconfig} version 0.15 or later (usually part of Gnome development packages).
\item {\tt groff} 1.10 or later (usually part of system packages).
\item {\tt Tcl/Tk} 8.0 or later development files (for eltclsh)
\end{itemize}

To edit /GenoM/ modules it is also recommended to use {\tt xemacs} to take
advantage of {\tt genom-mode}.

\section{What to download ?}
\label{sec|configuration|download}

\subsection{OpenRobots tools and libraries}

/GenoM/ is part of a suite of open-sources tools. 

In order to install and use /GenoM/ you will have to download and install
the following libraries and tools from  {\tt
http://softs.laas.fr/openrobots/} :

\begin{itemize}
\item {\tt mkdep}: LAAS tool to determine dependencies
\item {\tt pocolibs}: system communication and real-time primitives
\item {\tt libedit} (editline): generic line editing and history
functions (for eltclsh)
\item {\tt eltclsh}: interactive TCL shell (optional but very practical)
\item {\tt /GenoM/}: the generator of modules
\end{itemize}

Later on you can also be interested in the following opensource softwares
that are part of the OpenRobots suite and work nicely with /GenoM/:

\begin{itemize}
\item {\tt GDHE}: a tool to design 3D interface
\item {\tt OpenPRS}: a tool to design complexe supervisors
\end{itemize}

\subsubsection{mkdep}

{\tt Mkdep} is a tool to manage dependencies for make(1) automatically on
Unix-like systems.  The original feature of this version is to be able to
handle incremental updates of the dependencies.  The current release
version is 2.4 released on August 29, 2005.

It is recommended to install it.

\subsubsection{pocolibs}

{\tt Pocolib} is a system communication and real-time primitive layers used by
/GenoM/ modules. 

These libraries run on systems with POSIX.1 threads and basic real-time
extensions. It has been tested successfully on Solaris (7and above),
Linux (with glibc2), and NetBSD (2.0 and later). 

There is also some code to support RTAI and LXRT 3.0, but it is currently
untested and unsupported (this may change again in the future) 

\subsubsection{libedit / editline}

The {\tt editline} library provides generic line editing and history
functions, similar to those found in tcsh(1). This package contains a
Makefile for compiling the -current version of NetBSD's library which
provides the same consistent installation and compilation environments as
for the other tools found in this repository. 

Editline is used by eltclsh.

\subsubsection{eltclsh}



\section{Installation}
\label{sec|configuration|installation}

To install all these softwares, you will have to follow always the same
usual configuration/compilation/installation sequence. Here is the
example pocolibs. 

In this example, the environment variable {\tt PREFIX} allows to specify
the root path where the binaries, the libraries, the include files, and
so on will be installed. Typically you can set this variable to {\tt
/usr/local/openrobots}.

\begin{cartouche}
\begin{verbatim}
tar xfvz pocolibs-XYZ.tar.gz
cd pocolibs-XYZ
./autogen.sh
mkdir build
cd build
../configure --prefix=$PREFIX
make
make install
\end{verbatim}
\end{cartouche}

The command {\tt ./autogen.sh} is not always necessary. If there is
already a script {\tt configure} you can try it directly.

The {\tt configure} command has many options. You can see them with the
option {\tt --help}. For /eltclsh/ and /GenoM/

% -----------------------------------------------------------------------
\section{Configuration}
\label{sec|configuration|configuration}


