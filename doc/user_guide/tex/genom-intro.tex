%
% Copyright (c) 2001 LAAS/CNRS                        --  Thu Oct 18 2001
% All rights reserved.                                     Anthony Mallet
%
% This document is a translation of the French documentation of GenoM,
% originally written by Sara Fleury and Matthieu Herrb.
%
% Redistribution  and  use in source   and binary forms,  with or without
% modification, are permitted provided that  the following conditions are
% met:
%
%   1. Redistributions  of  source code must  retain  the above copyright
%      notice, this list of conditions and the following disclaimer.
%   2. Redistributions in binary form must  reproduce the above copyright
%      notice,  this list of  conditions and  the following disclaimer in
%      the  documentation   and/or  other  materials   provided with  the
%      distribution.
%
% THIS SOFTWARE IS PROVIDED BY THE  AUTHOR AND CONTRIBUTORS ``AS IS'' AND
% ANY  EXPRESS OR IMPLIED WARRANTIES, INCLUDING,  BUT NOT LIMITED TO, THE
% IMPLIED WARRANTIES   OF MERCHANTABILITY AND  FITNESS  FOR  A PARTICULAR
% PURPOSE ARE DISCLAIMED.  IN NO  EVENT SHALL THE AUTHOR OR  CONTRIBUTORS
% BE LIABLE FOR ANY DIRECT, INDIRECT,  INCIDENTAL, SPECIAL, EXEMPLARY, OR
% CONSEQUENTIAL DAMAGES (INCLUDING,  BUT  NOT LIMITED TO, PROCUREMENT  OF
% SUBSTITUTE  GOODS OR SERVICES;  LOSS   OF  USE,  DATA, OR PROFITS;   OR
% BUSINESS  INTERRUPTION) HOWEVER CAUSED AND  ON ANY THEORY OF LIABILITY,
% WHETHER IN CONTRACT, STRICT LIABILITY, OR TORT (INCLUDING NEGLIGENCE OR
% OTHERWISE) ARISING IN ANY WAY OUT OF THE  USE OF THIS SOFTWARE, EVEN IF
% ADVISED OF THE POSSIBILITY OF SUCH DAMAGE.
%

\GenoM\ (\textbf{Gen}erator \textbf{o}f \textbf{M}odules) is a development
framework that allows  the definition and the  production of modules that
encapsulate algorithms.  A module is  a standardized software entity that
is able to offer services which are provided by your algorithms.  Modules
can start or stop the execution of these services,  pass arguments to the
algorithms and export the data produced.

Now  you  might ask  yourself:  ``why should  I bother  integrating  into
modules my   own algorithms that   \emph{do} work  very  well?''. That's a
pretty good  question and this introduction  will try to advocate on that
point and give you some answers.

\bigbreak

Your  algorithms aim to  being embedded into  a  \emph{target} machine ---
let's say: a robot.  You might not embrace the way  this machine works in
its whole and, most important, your  algorithms will be integrated into a
more general software system that includes  other algorithms developed by
other persons.  This set of  algorithms  share several common properties:
they must be configured (don't you have a bunch of parameters you want to
adjust?), they must be started, interrupted, started again or stopped and
we might expect them to exchange data and communicate  with other part of
the system.

Consider the example of a mobile  robot: depending on the requirements of
its mission and the current context of execution, the robot might need to
acquire an image, localize itself,  build a local  map with some  sensors
and  move.  If the  environment  is rather free, the  robot  could plan a
trajectory, but  it could also  decide to move on the  basis of the local
data given by its  proximetric  sensors.  To be   able to schedule  these
rather complex (and uncertain!) actions, it  is \emph{necessary} to define
a protocol that can handle tasks at an abstract level. This protocol will
let the robot:

\begin{itemize}
\item start an action when it is needed;
\item stop an action in a clean manner;
\item pass a set of parameters and data to the action;
\item coordinate several actions;
\item get the results of these actions;
\item handle the failures of the actions (yes, actions can fail!) an keep
them from  taking the whole system with them when they spiral
down. Failures can be as general as:
   \begin{itemize}
   \item low batteries,
   \item incorrect algorithm parameters,
   \item not enough memory to handle that case,
   \item the target to localize does not appear in the images,
   \item the algorithm cannot handle that situation yet,
   \item \ldots
   \end{itemize}
\end{itemize}

Therefore,  the  general   concept  of  \emph{module}  and   \emph{standard
protocols}  have  been defined.   These  generic  modules can encapsulate
almost  every  kind of  algorithm:  periodic or aperiodic, synchronous or
asynchronous, interruptible or uninterruptible, and even yours!

Of course, you could yourself write a module on the basis of that generic
model. But that's a long and difficult story: you will  have to port your
software on the different systems you want it to run on, you will have to
write  test procedures to  check that  your  module behaves correctly  in
every situation, \ldots\ and \GenoM\ already does it for you!

\bigbreak

The generator  of modules comes  with  a description \emph{language},  and
standard templates.  The templates will let you describe your module, the
services  it  can offer, and  for   each  service the  list  of  expected
parameters,  the algorithms (yours!) that will  be  executed, the results
along with their description, the failure messages and a few other items.

With the template file and the code of your algorithms --- sorry, you still
must write it yourself --- \GenoM\ produces:

\begin{itemize}
   \item \emph{a complete module} that can run on several flavor of Unix
or VxWorks, 
   \item \emph{interface libraries} that will let you use the services of
the module and get their results back,
   \item \emph{an interactive test program} that let you send several
requests to the module and trigger the execution of the corresponding
services.
\end{itemize}

\bigbreak

Now that you  have an idea  of what \GenoM\ can  be used for, this manual
will explain you \emph{how} to actually do it. You will learn

\begin{center}\begin{cartouche}\parbox{0.9\hsize}{
\parskip10pt
    1. \textbf{How to produce and use a first test module}, with a concrete
example.

    2. \textbf{How the generic modules work}, and \textbf{how they are
structured}. In particular, some useful vocabulary is explained.

    3. \textbf{How to describe your own modules}. The complete
specification language will be explained.

    4. \textbf{How to generate your modules}.

    5. \textbf{How to integrate your algorithms into the modules}.

    6. \textbf{How to use modules} (from an interactive program, from
another module, \ldots).
}\end{cartouche}\end{center}

