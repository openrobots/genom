%
% Copyright (c) 2001 LAAS/CNRS                        --  Wed Oct 31 2001
% All rights reserved.                                     Anthony Mallet
%
% This document is a translation of the French documentation of GenoM,
% originally written by Sara Fleury and Matthieu Herrb.
%
% Redistribution  and  use in source   and binary forms,  with or without
% modification, are permitted provided that  the following conditions are
% met:
%
%   1. Redistributions  of  source code must  retain  the above copyright
%      notice, this list of conditions and the following disclaimer.
%   2. Redistributions in binary form must  reproduce the above copyright
%      notice,  this list of  conditions and  the following disclaimer in
%      the  documentation   and/or  other  materials   provided with  the
%      distribution.
%
% THIS SOFTWARE IS PROVIDED BY THE  AUTHOR AND CONTRIBUTORS ``AS IS'' AND
% ANY  EXPRESS OR IMPLIED WARRANTIES, INCLUDING,  BUT NOT LIMITED TO, THE
% IMPLIED WARRANTIES   OF MERCHANTABILITY AND  FITNESS  FOR  A PARTICULAR
% PURPOSE ARE DISCLAIMED.  IN NO  EVENT SHALL THE AUTHOR OR  CONTRIBUTORS
% BE LIABLE FOR ANY DIRECT, INDIRECT,  INCIDENTAL, SPECIAL, EXEMPLARY, OR
% CONSEQUENTIAL DAMAGES (INCLUDING,  BUT  NOT LIMITED TO, PROCUREMENT  OF
% SUBSTITUTE  GOODS OR SERVICES;  LOSS   OF  USE,  DATA, OR PROFITS;   OR
% BUSINESS  INTERRUPTION) HOWEVER CAUSED AND  ON ANY THEORY OF LIABILITY,
% WHETHER IN CONTRACT, STRICT LIABILITY, OR TORT (INCLUDING NEGLIGENCE OR
% OTHERWISE) ARISING IN ANY WAY OUT OF THE  USE OF THIS SOFTWARE, EVEN IF
% ADVISED OF THE POSSIBILITY OF SUCH DAMAGE.
%

% =======================================================================
\section{The \texttt{genom} command}

\begin{description}

\item\textbf{Synopsis}

\texttt{genom [-Ocdinxyto] [-I\emph{path}] [-D\emph{macro}] <module>[.gen]}

\item\textbf{Description}

\texttt{genom}  is the  command that  generates a  module. The  \texttt{module}
argument is the name  of the file which  contains the description of  the
module. The \texttt{.gen} is   optional and  is  automatically appended  if
omitted.

\texttt{genom} accepts the following options:

\begin{tabularx}{\linewidth}{lX}
\texttt{-O} & \emph{Accept \textbf{O}bsolete syntax:} the syntax of
\texttt{.gen} files has changed a bit over time. This option tells
genom to still accept input syntax that is now considered as obsolete.
Warning: this option can produce output code that is incompatible with
what is generated without it. Use at your own risk. \\

\texttt{-i} &  \emph{Installs the  directory  \texttt{codels/}:} you should  use
this option when  generating the module  for the first time. When  called
with \texttt{-i}, \texttt{genom} will install  the directory \texttt{codels/} as
well as templates for  the codel files in this  directory.  It  will also
create  all the \texttt{Makefiles}  (in the main directory   and in the 
\texttt{codels} directory) used to compile  the module.  If  the files that would
normally be  installed  are already present, \texttt{genom}  will ask  for
confirmation before overwriting files.\\

\texttt{-c} & \emph{Conditional regeneration:} module is regenerated only
if files from which  it depends have  been modified since last generation
(\emph{i.e.} the \texttt{.gen} file or files it includes).\\

\texttt{-t} &    \emph{Tcl libraries  generation:}    Generates Tcl interface
libraries. They are mandatory  if you wish to  control this module from a
tcl interpreter (see section~\ref{sec|tcl}).\\

\texttt{-o}  & \emph{OpenPRS libraries generation:} Generates OpenPRS
interface libraries.  They are  mandatory  if you   wish to  control this
module from an OpenPRS program.\\

\texttt{-x} & \emph{Tclserv C Client generation:} Generates Tclserv C interface
libraries. They can be used to control this module in C from a remote host, using
Tclserv protocol.\\

\texttt{-y} & \emph{Python client generation:} Generate Python interface (using
ctypes) to manipulate the different GeNoM structure. They can be used further to
create Python controller, or to directly access to posters. \\

\texttt{-n} & Generates the perl script that is used to generate the module,
without actually executing it.\\

\texttt{-d} & Turns on debugging mode inside the \emph{yacc} parser.\\

\texttt{-I\emph{path}} & Defines paths for included files (same as 
\texttt{-I} option of \texttt{cc}).\\

\texttt{-D\emph{macro}} & Defines macros as for \texttt{cc}.\\
\end{tabularx}

Once the module has been installed (\texttt{-i} option)  for the first time,
you just   have to invoke \texttt{make}   (GNUmake is required)  in the main
directory to regenerate or compile the module.

The \texttt{-i} option modifies files in the \texttt{codels/} directory. Use it
carefully.  If you wish  to manually  get a  new   template file for  the
codels, you can find them in the directory \texttt{.genom/codels/}.  These files
are always  up-to-date. Also consider the  XEmacs  mode \texttt{genom-mode},
which lets  you  insert codel templates  into  existing codel files  (see
chapter~\ref{cha|edit}).

\end{description}


% =======================================================================
\section{Product of the generation}

The module generation produces files in the two directories \texttt{codels/}
and \texttt{server/}.  The files located  in the  \texttt{codels/}  directory are
described in the next chapter.  This  section describes the files located
in  the  \texttt{server/} directory.   The \texttt{demo} module   is  used as an
example: you can always replace the string \texttt{demo} by the name of your
module.

Compiling these files produces binary objects and executables in the 
\texttt{\$\{TARGET\}} sub-\-directories. See appendix~\ref{anx|files} for
information on the file-system hierarchy.

% -----------------------------------------------------------------------
\subsection{Interface libraries}

\subsubsection{Requests library: \texttt{demoMsgLib}}

This library implements  the basic requests  functions (request emission,
replies reception) and is used by the clients of this module.

The \texttt{C}  source code    of  this  library  can be   found  in  
\texttt{demoMsgLib.c}.  The header
file \texttt{demoMsgLib.h}, which must  be included by clients, contains the
definitions for the server identification and communication establishment
(name and size of the mailbox, function prototypes, \ldots).

To use this  library, you must link your  program with  
\texttt{-ldemoClient}. 


\subsubsection{Posters library: \texttt{demoPosterLib}}

This library implements the basic posters functions for this module (read
and display   functions).  The same    structure as above  is used:  
\texttt{demoPosterLib.c} is  the  source code. 
\texttt{demoPosterLib.h}   defines the name  of  the
posters along  with  their data structures and prototypes of access
functions.
The  \emph{cIDS} structure is
also defined there\footnote{the structures included by the \emph{cIDS} are
defined in the file \texttt{modules.h}, located in the genom source tree.}.

To  use this library,  you must link  with  \texttt{
-ldemoClient}.


% -----------------------------------------------------------------------
\subsection{Useful header files}

The file \texttt{demoHeader.h} must  be included  in every  codel file.  It
contains the definitions of several constants that characterize the module
(name, period, posters, \ldots)  as well as the two  macros that let  you
access the \emph{IDS} (\emph{SDI\_F} and \emph{SDI\_C}).

The file  \texttt{demoError.h} contains the  definitions of the  error codes
(reports declared in the \texttt{fail\_reports} field) of this module.

The file \texttt{demoType.h} defines the actual \emph{IDS} structure. It is
not necessarily the same as  in the  \texttt{.gen}  file, especially if  the
module   defines   reentrant   requests   (\emph{i.e.}   compatible  with
themselves).

% -----------------------------------------------------------------------
\subsection{Tcl library}

The files  \texttt{demoTcl.c} and \texttt{demo.tcl}  are used by Tcl to control
the module. See section~\ref{sec|tcl} in this document.


% -----------------------------------------------------------------------
\subsection{Openprs library}

The files located in the \texttt{openprs/} directory are  used by OpenPRS to
control the module. See section~\ref{sec|propice} in this document.

% -----------------------------------------------------------------------

\subsection{TclservClient Library}

The files located in the \texttt{tclservClient/} directory can be used by a C
client to remotely control the module. To use this lib, you need to link with 
\texttt{-ldemoTclservClient}. See section ~\ref{sec|tclserv_client_lib}
in this document for more details.

% -----------------------------------------------------------------------
\subsection{Executables: server, test program, initialization request}

The executable \texttt{demoTest} is an interactive test program. To use it,
you just have to  run
\texttt{\$\{TARGET\}/demoTest}.

The  files    \texttt{demoCntrlTask.c},   \texttt{demoMotionTask.c}    and 
\texttt{demoModuleInit.c} contain the module's execution tasks source code.  Once
compiled, they produce the  files  
\texttt{libdemoServer.a}.  The   file  \texttt{demoModuleInit.c} produces    the
function \texttt{demoTaskInit} which spawns the module.

The file \texttt{demoInit.c} contains the code that invokes the {\em
Initialization} request of the module (if the module defines one). To use
it, run \texttt{demoInit} (the executable is in 
\texttt{\$\{prefix\}/bin}.


% -----------------------------------------------------------------------
\subsection{Other files}

The  files \texttt{demoScan.h} and \texttt{demoPrint.h} contain the definition
of  the interactive   functions  that scan  the  arguments  of the module
requests. These functions are used by the test program \texttt{demoTest}.

Appendix~\ref{anx|files} gives an exhaustive list  of the files  produced
by \GenoM.
