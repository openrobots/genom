%
% Copyright (c) 2001 LAAS/CNRS                        --  Thu Nov  8 2001
% All rights reserved.                                     Anthony Mallet
%
% This document is a translation of the French documentation of GenoM,
% originally written by Sara Fleury and Matthieu Herrb.
%
% Redistribution  and  use in source   and binary forms,  with or without
% modification, are permitted provided that  the following conditions are
% met:
%
%   1. Redistributions  of  source code must  retain  the above copyright
%      notice, this list of conditions and the following disclaimer.
%   2. Redistributions in binary form must  reproduce the above copyright
%      notice,  this list of  conditions and  the following disclaimer in
%      the  documentation   and/or  other  materials   provided with  the
%      distribution.
%
% THIS SOFTWARE IS PROVIDED BY THE  AUTHOR AND CONTRIBUTORS ``AS IS'' AND
% ANY  EXPRESS OR IMPLIED WARRANTIES, INCLUDING,  BUT NOT LIMITED TO, THE
% IMPLIED WARRANTIES   OF MERCHANTABILITY AND  FITNESS  FOR  A PARTICULAR
% PURPOSE ARE DISCLAIMED.  IN NO  EVENT SHALL THE AUTHOR OR  CONTRIBUTORS
% BE LIABLE FOR ANY DIRECT, INDIRECT,  INCIDENTAL, SPECIAL, EXEMPLARY, OR
% CONSEQUENTIAL DAMAGES (INCLUDING,  BUT  NOT LIMITED TO, PROCUREMENT  OF
% SUBSTITUTE  GOODS OR SERVICES;  LOSS   OF  USE,  DATA, OR PROFITS;   OR
% BUSINESS  INTERRUPTION) HOWEVER CAUSED AND  ON ANY THEORY OF LIABILITY,
% WHETHER IN CONTRACT, STRICT LIABILITY, OR TORT (INCLUDING NEGLIGENCE OR
% OTHERWISE) ARISING IN ANY WAY OUT OF THE  USE OF THIS SOFTWARE, EVEN IF
% ADVISED OF THE POSSIBILITY OF SUCH DAMAGE.
%
% $Id$
%

\section{Module generation}

Avoid conflicts with \GenoM\ keywords: in the \texttt{.gen} file, do not use
variables nor functions named \texttt{type}, \texttt{control}, \texttt{poster},
\ldots

Avoid conflicts with structure names which are generated by \GenoM: 
\texttt{DEMO\_STR}, \ldots

\GenoM\ should parse every valid \texttt{C} file. As of today, a few
problems remain, especially for \texttt{union}s and  recursive typedefs such as
\texttt{typedef  PILO\_MOVE\_STR PILO\_MOVE\_STR\_2[2]}.  You should not use
them at this time.

Problems may also arise if you use different names for  a structure and a
typedef associated to that structure, as in the following example:

\begin{center}\begin{cartouche}\small\begin{verbatim}
/* XXX avoid that at this time */
typedef struct DIST_STR {
   double dist;
} dist_str;
\end{verbatim}\end{cartouche}\end{center}


\section{Execution under Unix}

\subsection{\texttt{csLib} initialization failures}

\begin{center}\begin{cartouche}\small\begin{verbatim}
blues% h2 init
Initializing csLib devices: 
Cslib devices already exist on this machine.
Do you want to delete and recreate them (y/n) ? 
\end{verbatim}\end{cartouche}\end{center}

$\rightarrow$ \texttt{csLib} is already initialized.  You can answer \texttt{n}
if  everything is  ok  and  you  don't   need to initialize  \texttt{csLib}
again. Answer \texttt{y} if you need to reset \texttt{csLib}.


\subsection{Module startup failures}

\begin{center}\begin{cartouche}\small\begin{verbatim}
blues% ./codels/i386-linux/demo -b
Hilare2 execution environment version 2.11
Copyright (c) 1999-2011 LAAS/CNRS
demo: error creating /home/matthieu/.demo.pid: File exists
\end{verbatim}\end{cartouche}\end{center}

$\rightarrow$ You didn't kill properly an old instance of the module. Use
the command ``\texttt{killmodule <module-name>}''.


\begin{center}\begin{cartouche}\small\begin{verbatim}
waits[demo] ./codels/sparc-solaris/demo
DEMO :
Spawn control task ... demoCntrlTask/posterCreate:
                                              S_h2devLib_DUPLICATE_DEVICE_NAME
\end{verbatim}\end{cartouche}\end{center}

$\rightarrow$ The  control poster already  exists: an  old module was not
killed properly, see above.


\subsection{Interactive ``Test'' program failures}

The shell  blocks   after sending  a  request: launch  another  task with
another number.



\bigbreak
\textbf{Common bugs in codels}

Writes beyond  arrays'  bounds and  writes  in memory  pages  that do not
belong to you are  very common mistakes.   Use common Unix tools such  as
\texttt{purify} or \texttt{workshop} to debug.
