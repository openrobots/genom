%
% Copyright (c) 2001 LAAS/CNRS                        --  Thu Nov  8 2001
% All rights reserved.                                     Anthony Mallet
%
% This document is a translation of the French documentation of GenoM,
% originally written by Sara Fleury and Matthieu Herrb.
%
% Redistribution  and  use in source   and binary forms,  with or without
% modification, are permitted provided that  the following conditions are
% met:
%
%   1. Redistributions  of  source code must  retain  the above copyright
%      notice, this list of conditions and the following disclaimer.
%   2. Redistributions in binary form must  reproduce the above copyright
%      notice,  this list of  conditions and  the following disclaimer in
%      the  documentation   and/or  other  materials   provided with  the
%      distribution.
%
% THIS SOFTWARE IS PROVIDED BY THE  AUTHOR AND CONTRIBUTORS ``AS IS'' AND
% ANY  EXPRESS OR IMPLIED WARRANTIES, INCLUDING,  BUT NOT LIMITED TO, THE
% IMPLIED WARRANTIES   OF MERCHANTABILITY AND  FITNESS  FOR  A PARTICULAR
% PURPOSE ARE DISCLAIMED.  IN NO  EVENT SHALL THE AUTHOR OR  CONTRIBUTORS
% BE LIABLE FOR ANY DIRECT, INDIRECT,  INCIDENTAL, SPECIAL, EXEMPLARY, OR
% CONSEQUENTIAL DAMAGES (INCLUDING,  BUT  NOT LIMITED TO, PROCUREMENT  OF
% SUBSTITUTE  GOODS OR SERVICES;  LOSS   OF  USE,  DATA, OR PROFITS;   OR
% BUSINESS  INTERRUPTION) HOWEVER CAUSED AND  ON ANY THEORY OF LIABILITY,
% WHETHER IN CONTRACT, STRICT LIABILITY, OR TORT (INCLUDING NEGLIGENCE OR
% OTHERWISE) ARISING IN ANY WAY OUT OF THE  USE OF THIS SOFTWARE, EVEN IF
% ADVISED OF THE POSSIBILITY OF SUCH DAMAGE.
%
% $Id$
%

\section{Module generation}

Avoid conflicts with \GenoM\ keywords: in the {\tt .gen} file, do not use
variables nor functions named {\tt type}, {\tt control}, {\tt poster},
\ldots

Avoid conflicts with structure names which are generated by \GenoM: {\tt
DEMO\_STR}, \ldots

\GenoM\ should parse every valid {\tt C} file. As of today, a few
problems remain, especially for {\tt union}s and  recursive typedefs such as
{\tt typedef  PILO\_MOVE\_STR PILO\_MOVE\_STR\_2[2]}.  You should not use
them at this time.

Problems may also arise if you use different names for  a structure and a
typedef associated to that structure, as in the following example:

\begin{center}\begin{cartouche}\small\begin{verbatim}
/* XXX avoid that at this time */
typedef struct DIST_STR {
   double dist;
} dist_str;
\end{verbatim}\end{cartouche}\end{center}


\section{Execution under Unix}

\subsection{{\tt csLib} initialization failures}

\begin{center}\begin{cartouche}\small\begin{verbatim}
rantanplan% h2 init
Initializing csLib devices: 
Cslib devices already exist on this machine.
Do you want to delete and recreate them (y/n) ? 
\end{verbatim}\end{cartouche}\end{center}

$\rightarrow$ {\tt csLib} is already initialized.  You can answer {\tt n}
if  everything is  ok  and  you  don't   need to initialize  {\tt  csLib}
again. Answer {\tt y} if you need to reset {\tt csLib}.


\subsection{Module startup failures}

\begin{center}\begin{cartouche}\small\begin{verbatim}
rantanplan% ./codels/i386-linux/demo -b
Hilare2 execution environment version 1.0
Copyright (c) 1999 LAAS/CNRS
demo: error creating /home/matthieu/.demo.pid: File exists
\end{verbatim}\end{cartouche}\end{center}

$\rightarrow$ You didn't kill properly an old instance of the module. Use
the command ``{\tt killmodule <module-name>}''.


\begin{center}\begin{cartouche}\small\begin{verbatim}
waits[demo] ./codels/sparc-solaris/demo
DEMO :
Spawn control task ... demoCntrlTask/posterCreate:
                                              S_h2devLib_DUPLICATE_DEVICE_NAME
\end{verbatim}\end{cartouche}\end{center}

$\rightarrow$ The  control poster already  exists: an  old module was not
killed properly, see above.


\subsection{Interactive ``Essay'' program failures}

The shell  blocks   after sending  a  request: launch  another  task with
another number.


% =======================================================================
\section{VxWorks}

\subsection{Loading failures}

\begin{center}\begin{cartouche}\small\begin{verbatim}
undefined symbol: rdima
\end{verbatim}\end{cartouche}\end{center}

$\rightarrow$ The dynamic link edition failed: the  symbol {\tt rdima} (a
function, or  a variable) is undefined. You   must load the  library that
defines it before the library that uses it.


\subsection{Module startup failures}

\subsection{Interactive ``Essay'' program failures}

The terminal window does not show up.
\begin{itemize}
\item You didn't launch {\tt xes\_server} on the remote station.
\item The {\tt DISPLAY} variable is unset, or is not properly set
\item You do not authorize the remote machine to display on your display
server. 
\item You did not define under VxWorks which Unix machine will host the
display:\\ use {\tt   xes\_set\_host <host>} and {\tt xes\_get\_host}  to
check.
\end{itemize}

\subsection{It was working, but...}

Besides problems specific to your codels, some common problems can be the
cause of sudden failures:

\bigbreak
{\bf A stack is too small}

You can easily   detect this  with the  command  {\tt checkStack}.   This
command displays  the stack  usage  for all tasks.    If a stack  size is
labeled {\tt OVERFLOW}, the stack is too small. If the corresponding task
is an execution  task of a module, you   just have to increase its  size,
recompile and run the module again ({\tt stack\_size} field). If the task
is a control task, you have to  turn the control  codel into an execution
codel (and increase the stack size of the consequent execution task).

Local variables in your  functions are stored   on the stack.  Avoid  big
arrays  and  other huge   structures.  Be  careful   with the {\tt stdio}
functions such as {\tt fopen} which are great stack consumers!


\bigbreak
{\bf Not enough memory}

You can detect this with the command {\tt memShow}. Check that you do not
use huge data structures and that you free memory as often as possible.


\bigbreak
{\bf Common bugs in codels}

Writes beyond  arrays'  bounds and  writes  in memory  pages  that do not
belong to you are  very common mistakes.   Use common Unix tools such  as
{\tt purify} or {\tt workshop} to debug.
